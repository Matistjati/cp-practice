\problemname{Trollkarlen Thomas}
Enda sedan Thomas tog examen från trollkarshögskolan uppe i Kiruna har han försökt att utveckla en process för att transmutera guld. För att göra detta utför
han en serie \textit{experiment}. Varje experiment består av att han tar allt material av en viss typ och transmuterar de till ett annat, oförutsägbart valt
material. För att hålla koll på alla sina experiment har han gett varje material en unik siffra mellan $1$ och $10^9$. 

Nyligen har ett flertal örter förbjudits, däribland Skymningsnektar, Sirensång och ett flertal släktingar till oregano på grund av deras
hallucinogeniska egenskaper. Efter att ha genomfört otaliga experiment är Thomas orolig att något i hans källare är olagligt. Istället för att rota
igenom hela källaren vill han använda sina noggrana anteckningar. Varje gång han köpt ett material eller utfört ett experiment har han skrivit ner dessa.
Kan du hjälpa honom att räkna hur mycket av varje sak han har i sin källare? (Han slänger aldrig något...)

\section*{Indata}
Den första raden innehåller två heltal $N$ ($1 \leq N \leq 10^6$), antalet rader i hans lista av anteckningar.
Därefter följer $N$ rader vardera beskriver en rad i anteckningarna. Dessa kommer i kronologisk ordning.

Om den första siffran är $1$ följer heltalen $T,K$ ($1 \leq T, K \leq 10^9$), vilket betyder att Thomas köpt $K$ stycken enheter av material $T$.

Om den första siffran är $2$ följer heltalen $T_1, T_2$ ($1 \leq T_1, T_2 \leq 10^9$), vilket betyder att Thomas omvandlar allt material av typ $T_1$ till typ $T_2$.


\section*{Utdata}
För varje material som Thomas har i sin källare, skriv först ut materialets siffra och sedan hur många enheter av materialet han har.
Skriv ut dessa i storleksordning relativt materialens siffror.

\section*{Poängsättning}
Din lösning kommer att testas på flera testfall.
\noindent
För att lösa problemet måste din lösning lösa alla testfallen korrekt.

\noindent
\begin{tabular}{| l | l | l |}
\hline
  Grupp & Poängvärde & Gränser \\ \hline
  $1$    & $20$       &  Den högsta siffran som förekommer i indatan är $10^5$ \\ \hline
  $2$    & $20$       &  $N \leq 10^5$ \\ \hline
  $3$    & $30$       &  $N \leq 5 \cdot 10^5$ \\ \hline
  $4$    & $30$       &  Inga ytterligare begränsningar \\ \hline
\end{tabular}
