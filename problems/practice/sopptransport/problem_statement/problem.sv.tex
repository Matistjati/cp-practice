\problemname{Christians algoritm}
Christian fick precis lära sig om det klassiska "sopptransports-problemet". I detta klassiska problem har du en bil som kan transportera upp till $C$ kg soppa.
Du har ett lager med $N$ stycken soppburkar, där varje har en vikt $W$ och en kvalitet $K$. Målet är att planera nästa transport, där målet är att
välja ett par soppburkar så att summan av kvaliteten av soppan du transporterar är så stor som möjligt, samtidigt som summan av deras vikt är mindre än $C$ kg.

Det tar inte ens 5 sekunder för honom att komma på följande lösning: sortera listan först med avseende på $\frac{K}{W}$. D.v.s. så kommer en soppburk med $K=5,W=2$ före en soppburk
med $K=2,W=1$. Intuitivt ger dessa mest kvalitet per vikt. Därefter går vi igenom listan från vänster till höger och lastar på soppburkar så länge tilläget av denna soppburk inte
kommer orsaka att totalvikten av lastbilen överskrids.

Han insåg dock snabbt att denna algoritm inte alltid är optimal. Därför gjorde han följande förbättring: han kör ovanstående algoritm, fast testar även att starta den på
varje möjlig startpunkt. D.v.s., först testar han att köra den från början, därefter lastar han ut soppbilen och testar att istället starta om från position $2$ i listan av
soppburkar, och så vidare. Denna algoritm funkar så bra att han inte hittar fall där den ger fel svar. Kan du hjälpa honom att avgöra om hans algoritm ger bästa svaret
för en given instans av sopptransports-problemet?

\section*{Indata}
Den första raden innehåller heltalen $N, C$ ($1 \leq N \leq 5*10^4, 1 \leq C \leq 100$), antalet soppburkar och hur många kg som soppbilen kan transportera.
Därefter $N$ rader, där den $i$:te raden innehåller heltalen $K_i,W_i$ ($1 \leq K_i \leq 1000, 1 \leq W_i \leq C$), kvaliteten och vikten på den $i$:te soppburken.

\section*{Utdata}
Om Christians algoritm ger det optimala svaret, skriv ut "ja". Annars, skriv ut "nej".

\section*{Poängsättning}
Din lösning kommer att testas på en mängd testfallsgrupper.
För att få poäng för en grupp så måste du klara alla testfall i gruppen.

\noindent
\begin{tabular}{| l | l | p{12cm} |}
  \hline
  \textbf{Grupp} & \textbf{Poäng} & \textbf{Gränser} \\ \hline
  $1$    & $25$      & $ N \leq 1000$ \\ \hline
  $2$    & $25$      & Inga ytterligare begränsningar \\ \hline
\end{tabular}
