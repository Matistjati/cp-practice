\problemname{Range Sum 3}
\newcommand{\Mod}[1]{\ (\mathrm{mod}\ #1)}

Du har en lista med heltalen $A_0, A_1, \dots, A_{10^9}$. Till en början är $A_i=0$ för alla $i=0,1,\dots,10^9$.
Du ska hantera två sorters frågor:

\begin{itemize}
    \item Typ 0: givet heltalen $l$ och $r$, skriv ut värdet på  $A_l + A_{l+1} + \cdots + A_{r} \Mod{10^9+7}$
    \item Typ 1: givet heltalen $l$, $r$ och $v$, sätt $A_i=A_i+v$
\end{itemize}

\section*{Indata}
Den första raden innehåller heltalet $Q$ ($1 \leq Q \leq 10^5$).
Därefter följer $Q$ rader som innehåller en fråga vardera. Dessa ska behandlas som beskrivet ovan.
Varje fråga börjar med talet $T$, typen av frågan.

Om $T=0$ följer heltalen $l,r$ ($0 \leq l \leq r \leq 10^9$).

Om $T=1$ följer heltalen $l, r, v$ ($0 \leq l \leq r \leq 10^9$, $1 \leq V \leq 10^9$).

\section*{Utdata}
För varje fråga med $T=0$, skriv ut summan i det efterfrågade intervallet.

\section*{Poängsättning}
Din lösning kommer att testas på flera testfallsgrupper.
\noindent
För att få poäng för en grupp så måste du klara alla testfall i gruppen.

\noindent
\begin{tabular}{| l | l | l |}
\hline
  Grupp & Poängvärde & Gränser \\ \hline
  $1$    & $40$       &  $r \leq 10^5$ för alla frågor \\ \hline
  $2$    & $10$       &  $l=r$ för alla frågor \\ \hline
  $3$    & $10$       &  $l=r$ för alla frågor där $T=1$ \\ \hline
  $4$    & $40$       &  Inga ytterligare begränsningar \\ \hline
\end{tabular}
