\problemname{Sneetches}
There is a line of Sneetches on a beach. Each one may or may not have a star on its belly initially.

Your task is to process $Q$ queries:

\begin{itemize}
    \item Type 1: given integers $l, r$, each Sneetch in the range $[l,r]$ will swap whether or not it has a star on its belly.
    So if it had a star on its belly before, it doesn't after and vice versa.
    \item Type 2: given integers $l, r$, sort all Sneetches in the range $[l, r]$ such that all sneeches without a star
    come first, and then all the Sneetches with a star.
\end{itemize}

After each query, output the largest consecutive interval of Sneetches that have the same number of stars on their bellies.

\section*{Input}
The first line of input contains the integers $N,Q$ ($1 \leq N,Q \leq 2 \cdot 10^5$).

The second line contains a binary string of length $N$, the $i$:th character being $1$ symbolizing that the $i$:th sneetch has
a star on its belly, and vice versa. 

The following $Q$ lines each contain one query each. Each query is described by the integers
$T, l, r$ ($1 \leq T \leq 2$, $0 \leq l \leq r \leq N - 1$), the type and the bounds of the query.

\section*{Output}
After each query, output the size of the largest consecutive interval of Sneetches that have the same number of stars on their bellies.


\section*{Scoring}
Your solution will be tested on a set of test groups, each worth a number of points.
Each test group contains a set of test cases.
To get the points for a test group you need to solve all test cases in the test group.

\noindent
\begin{tabular}{| l | l | p{12cm} |}
  \hline
  \textbf{Group} & \textbf{Point value} & \textbf{Constraints} \\ \hline
  $1$    & $25$         & $T = 2$  \\ \hline
  $2$    & $50$         & $T = 1$ \\ \hline
  $3$    & $25$         & No additional constraints. \\ \hline
\end{tabular}

Note: this problem is an adaptation of the problem Sneetches and Speeches 3 by SecondThread, which is itself
an adaptation of Star-Belly Sneetches by Matt Fontaine. The use of this task is to have a task with only the
xor and sort operations, so that treaps aren't needed.

\section*{Explanation of sample}
Sample 1:
\begin{itemize}
  \item After query 1: 101101
  \item After query 2: 111101
  \item After query 3: 000011
  \item After query 4: 000000
\end{itemize}


Sample 2:
\begin{itemize}
  \item After query 1: 101100
  \item After query 2: 001110
  \item After query 3: 010001
  \item After query 4: 100001
  \item After query 5: 001001
\end{itemize}
