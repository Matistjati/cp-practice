\problemname{Djikstra, who?}
Givet är en oriktad graf med $N$ noder och $M$ kanter. Varje nod har en vikt. Du vill avgöra om det finns en väg
i grafen från nod $1$ till nod $N$ där summan av dess kantvikter inte är större än $10^7$. Om en sådan väg 
finns ska du skriva ut längden på den, skriv annars ut $-1$.

\section*{Indata}
Den första raden innehåller två heltal $N,M$ ($2 \leq N \leq 10^6$, $0 \leq M \leq 10^6$), antalet noder och antalet kanter i grafen.
Därefter följer $M$ rader som innehåller tre heltal $A,B,C$ ($1 \leq A,B \leq N$, $0 \leq C \leq 10^7$),
vilket betyder att det finns en kant mellan nod $A$ och nod $B$ med vikt $C$.

Det finns maximalt en kant mellan varje par av noder och inga kanter från en nod till sig själv.

\section*{Utdata}
Skriv ut ett heltal: om det finns en väg mellan $1$ och $N$ som inte är större än $10^7$, skriv ut längden på denna.
Annars, skriv ut $-1$.

\section*{Poängsättning}
Din lösning kommer att testas på flera testfall.
\noindent
För att lösa problemet måste din lösning lösa alla testfallen korrekt.
