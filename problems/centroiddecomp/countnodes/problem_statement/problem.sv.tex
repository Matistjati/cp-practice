\problemname{Noder i närheten}
Givet är ett träd med $N$ noder numrerade mellan $0$ och $N-1$. Du ska hantera följande fråga:

\begin{itemize}
    \item Givet noden $u$ och heltalet $d$, räkna antalet unika noder du kan nå om du startar på $u$ och korsar som mest $d$
    stycken kanter.
\end{itemize}


\section*{Indata}
Den första raden innehåller heltalen $N, Q$ ($1 \leq N,Q \leq 2*10^5$), antalet noder och antalet frågor.

De följande $N-1$ rader kommer vardera innehålla heltalen $a$ och $b$ ($0 \leq a,b \leq N-1$), vilket
betyder att det finns en kant mellan nod $a$ och $b$. Dessa kanter är garanterade att forma ett träd.

Därefter följer $Q$ rader som innehåller en fråga vardera. 

Varje fråga består av heltalen $u$ och $d$ ($0 \leq u,d \leq N-1)$, startnoden och antalet kanter du får korsa.

\section*{Utdata}
För varje fråga, skriv ut antalet unika noder du kan nå från $u$ om du korsar som mest $d$ kanter.

\section*{Poängsättning}
Din lösning kommer att testas på flera testfall.
\noindent
För att lösa problemet måste din lösning lösa alla testfallen korrekt.
